\chapter{Analisis}
\label{chap:analisis}

\section{Analisis Portal Akademik Mahasiswa}
Portal Akademik Mahasiswa merupakan sebuah situs jaringan yang diperuntukan bagi mahasiswa dalam rangka mendapatkan informasi kegiatan akademik\cite{BTI:2012}. Mahasiswa dapat mengakses Portal Akademik Mahasiswa melalui URL \url{https://studentportal.unpar.ac.id/}. Untuk mengakses Portal Akademik Mahasiswa, mahasiswa harus \textit{login} menggunakan akun email \textit{student}. Halaman \textit{login} Student Portal UNPAR terintegrasi dengan CAS (\textit{Central Authentication Service}) UNPAR\footnote{\url{https://cas.unpar.ac.id}}.

\begin{figure}[H]
	\centering
	\includegraphics[scale=0.5]{Gambar/pam-home}
	\caption{Halaman Utama Portal Akademik Mahasiswa} 
	\label{fig:3_pam_home}
\end{figure}

Pada halaman utama Portal Akademik Mahasiswa (gambar \ref{fig:3_pam_home}), terdapat beberapa bagian yaitu:
\begin{enumerate}
	\item Menu Atas\\
	Menu ini berfungsi sebagai menu pendukung yang terdiri dari : 
	\begin{itemize}
		\item \textbf{Home}, menampilkan informasi atau pengumuman yang dikeluarkan oleh fakultas masing-masing (Gambar \ref{fig:3_pam_atas_home}). 
		
		\begin{figure}[H]
			\centering
			\includegraphics[scale=0.5]{Gambar/pam-atas-home}
			\caption{Menu Atas Home} 
			\label{fig:3_pam_atas_home}
		\end{figure}
		
		\item \textbf{Kuliah}, menampilkan pengumuman per mata kuliah sesuai dengan mata kuliah dan kelas yang diambil oleh masing-masing mahasiswa (Gambar \ref{fig:3_pam_atas_kuliah}).  
		
		\begin{figure}[H]
			\centering
			\includegraphics[scale=0.5]{Gambar/pam-atas-kuliah}
			\caption{Menu Atas Kuliah} 
			\label{fig:3_pam_atas_kuliah}
		\end{figure}
		
		\item \textbf{Profil}, berisi tentang data diri masing-masing mahasiswa (Gambar \ref{fig:3_pam_atas_profil}). 
		
		\begin{figure}[H]
			\centering
			\includegraphics[scale=0.5]{Gambar/pam-atas-profil}
			\caption{Menu Atas Profil} 
			\label{fig:3_pam_atas_profil}
		\end{figure}
		
		\item \textbf{Komentar}, berisi komentar, saran, dan kritik dari mahasiswa (Gambar \ref{fig:3_pam_atas_komentar}).
		
		\begin{figure}[H]
			\centering
			\includegraphics[scale=0.5]{Gambar/pam-atas-komentar}
			\caption{Menu Atas Komentar} 
			\label{fig:3_pam_atas_komentar}
		\end{figure}

	\end{itemize}
	
	\item Identitas Portal \\
	Bagian ini menampilkan identitas pengguna portal. Tampilan identitas ini dapat ditampilkan lengkap dengan melakukan klik pada \textit{link} ``selengkapnya'' atau ditampilkan minimal dengan klik \textit{link} ``tutup''. Identitas yang ditampilkan adalah nama, Nomor Pokok Mahasiswa (NPM), status keaktifan, pas foto, email, dosen wali, program studi, dan fakultas seperti yang terlihat pada gambar \ref{fig:3_pam_identitas}.   
	\begin{figure}[H]
			\centering
			\includegraphics[scale=0.75]{Gambar/pam-identitas}
			\caption{Identitas Portal} 
			\label{fig:3_pam_identitas}
		\end{figure}
		
	\item Menu Utama\\
	Bagian ini memuat fitur utama Portal Akademik Mahasiswa mengenai data akademik (gambar \ref{fig:3_pam_utama}) yang terdiri dari:
		\begin{figure}[H]
			\centering
			\includegraphics[scale=0.75]{Gambar/pam-utama}
			\caption{Menu Utama} 
			\label{fig:3_pam_utama}
		\end{figure}
	\begin{itemize}
	
		\item \textbf{Rencana Studi}\\
		Menu Rencana Studi terdiri dari submenu: 
		\begin{itemize}
			\item Registrasi (FRS/PRS)\\
			Digunakan sebagai formulir pengisian rencana studi awal (FRS) dan perubahan rencana studi (PRS) (Gambar \ref{fig:3_pam_utama_registrasi}). 			%belum relevan
			\begin{figure}[H]
				\centering
				\includegraphics[scale=0.5]{Gambar/pam-utama-rencanastudi}
				\caption{Tampilan Registrasi FRS/PRS} 
				\label{fig:3_pam_utama_registrasi}
			\end{figure}
			
			\item Kartu Rencana Studi \\
			Menampilkan informasi mata kuliah yang telah diambil melalui submenu Registrasi (Gambar \ref{fig:3_pam_utama_krs}). Kartu Rencana Studi juga dapat dicetak melalui submenu ini. 
			\begin{figure}[H]
				\centering
				\includegraphics[scale=0.5]{Gambar/pam-utama-krs}
				\caption{Tampilan Kartu Rencana Studi} 
				\label{fig:3_pam_utama_krs}
			\end{figure}
			
			\item Pindah Kelas MKU \\
			Mahasiswa dapat memilih kelas yang masih tersedia di kolom Jadwal Baru dan menekan tombol ``Simpan'' untuk setiap kelas yang diubah (Gambar \ref{fig:3_pam_utama_pindahmku}). 
			%belum relevan
			\begin{figure}[H]
				\centering
				\includegraphics[scale=0.5]{Gambar/pam-utama-pindahmku}
				\caption{Tampilan Pindah Kelas MKU} 
				\label{fig:3_pam_utama_pindahmku}
			\end{figure}

		\end{itemize}
		
		\item \textbf{ Jadwal}\\
		Menu Jadwal terdiri dari submenu: 
		\begin{itemize}
			\item Kuliah, UTS, dan UAS \\
			Submenu ini berisi tentang jadwal kuliah, UTS dan UAS yang dapat disusun per semester (Gambar \ref{fig:3_pam_utama_jadwal}). 
			\begin{figure}[H]
				\centering
				\includegraphics[scale=0.5]{Gambar/pam-utama-jadwal}
				\caption{Tampilan Jadwal Kuliah, UTS, dan UAS} 
				\label{fig:3_pam_utama_jadwal}
			\end{figure}
			
			\item MKU \\
			Submenu ini menampilkan seluruh jadwal Mata Kuliah Umum (MKU) yang memberikan informasi tentang kelas-kelas yang dibuka oleh Pusat Kajian Humaniora (PKH) (Gambar \ref{fig:3_pam_utama_jadwalmku}). 
			\begin{figure}[H]
				\centering
				\includegraphics[scale=0.5]{Gambar/pam-utama-jadwalmku}
				\caption{Tampilan Jadwal MKU} 
				\label{fig:3_pam_utama_jadwalmku}
			\end{figure}
			
			\item Seluruh Fakultas \\
			Fitur ini memberikan informasi mengenai jadwal-jadwal yang ada di seluruh fakultas (Gambar \ref{fig:3_pam_utama_jadwalall}).
			\begin{figure}[H]
				\centering
				\includegraphics[scale=0.5]{Gambar/pam-utama-jadwalall}
				\caption{Tampilan Jadwal Seluruh Fakultas} 
				\label{fig:3_pam_utama_jadwalall}
			\end{figure}
		\end{itemize}
		
		\item \textbf{Nilai dan Indeks Prestasi}\\
		Menu Nilai dan Indeks Prestasi terdiri dari submenu: 
		\begin{itemize}
			\item Riwayat per Semester \\
			Submenu ini menampilkan informasi nilai per semester. Mahasiswa dapat melihat nilai sesuai dengan semester yang dipilih atau bisa memilih
pilihan ``Seluruh Tahun Akademik'' untuk melihat seluruh nilai berdasarkan semester (Gambar \ref{fig:3_pam_utama_nilai}).
			\begin{figure}[H]
				\centering
				\includegraphics[scale=0.5]{Gambar/pam-utama-nilai}
				\caption{Tampilan Riwayat Per Semester} 
				\label{fig:3_pam_utama_nilai}
			\end{figure}
			
			\item Daftar Perkembangan Studi \\
			Seluruh riwayat mata kuliah dan nilai yang pernah ditempuh ditampilkan di submenu ini (Gambar \ref{fig:3_pam_utama_dps}). Pada bagian bawah halaman, terdapat statistik nilai dan indeks prestasi (Gambar \ref{fig:3_pam_utama_dpsstat}). 
			\begin{figure}[H]
				\centering
				\includegraphics[scale=0.5]{Gambar/pam-utama-dps}
				\caption{Tampilan Daftar Perkembangan Studi} 
				\label{fig:3_pam_utama_dps}
			\end{figure}
			
			\begin{figure}[H]
				\centering
				\includegraphics[scale=0.5]{Gambar/pam-utama-dpsstat}
				\caption{Tampilan Statistik Nilai dan IP} 
				\label{fig:3_pam_utama_dpsstat}
			\end{figure}
			
			\item Riwayat Indeks Prestasi \\
			Menampilkan daftar riwayat indeks prestasi semester dan kumulatif setiap semester. Tampilan ini juga dilengkapi dengan grafik perkembangan (Gambar \ref{fig:3_pam_utama_ip}). 
			\begin{figure}[H]
				\centering
				\includegraphics[scale=0.5]{Gambar/pam-utama-ip}
				\caption{Tampilan Riwayat Indeks Prestasi} 
				\label{fig:3_pam_utama_ip}
			\end{figure}
			
			\item TOEFL \\
			Menampilkan daftar riwayat skor \textit{Test of English as Foreign Language} (TOEFL) yang pernah ditempuh (Gambar \ref{fig:3_pam_utama_toefl}). Mahasiswa diwajibkan untuk menempuh TOEFL dengan skor minimal 500.
			
			\begin{figure}[H]
				\centering
				\includegraphics[scale=0.5]{Gambar/pam-utama-toefl}
				\caption{Tampilan TOEFL} 
				\label{fig:3_pam_utama_toefl}
			\end{figure}
		\end{itemize}
		
		\item \textbf{Pembayaran Uang Kuliah}\\
		Menu ini berfungsi untuk melihat data tagihan pembayaran uang kuliah serta cara-cara pembayarannya (Gambar \ref{fig:3_pam_utama_pembayaran}).
		\begin{figure}[H]
				\centering
				\includegraphics[scale=0.5]{Gambar/pam-utama-pembayaran}
				\caption{Tampilan Pembayaran Uang Kuliah} 
				\label{fig:3_pam_utama_pembayaran}
			\end{figure}
		\end{itemize}
		
	\item \textbf{Informasi}\\
		Bagian ini menampilkan informasi tentang periode-periode yang sedang aktif (Gambar \ref{fig:3_pam_utama_informasi}). Sebagai contoh jika ``Periode Registrasi'' diklik maka akan muncul \textit{pop up} seperti pada gambar \ref{fig:3_pam_utama_informasipop}.
			\begin{figure}[H]
				\centering
				\includegraphics[scale=0.75]{Gambar/pam-utama-informasi}
				\caption{Tampilan Informasi} 
				\label{fig:3_pam_utama_informasi}
			\end{figure}
			
			\begin{figure}[H]
				\centering
				\includegraphics[scale=0.5]{Gambar/pam-utama-infopop}
				\caption{Tampilan \textit{Pop Up} Informasi} 
				\label{fig:3_pam_utama_informasipop}
			\end{figure}
		
	\item \textbf{Kalender}\\
		Bagian ini menampilkan kalender masehi (Gambar \ref{fig:3_pam_utama_kalender}).
		\begin{figure}[H]
				\centering
				\includegraphics[scale=0.75]{Gambar/pam-utama-kalender}
				\caption{Tampilan Kalender} 
				\label{fig:3_pam_utama_kalender}
			\end{figure}
		
	\item \textbf{Info Browser}\\
		Bagian ini menampilkan informasi tentang internet \textit{browser} yang digunakan pada saat membuka Portal Akademik Mahasiswa (Gambar \ref{fig:3_pam_utama_infobrowser}). 
		\begin{figure}[H]
				\centering
				\includegraphics[scale=0.75]{Gambar/pam-utama-infobrowser}
				\caption{Tampilan Info Browser} 
				\label{fig:3_pam_utama_infobrowser}
			\end{figure}
\end{enumerate}

\section{Analisis Kebutuhan IT Student Portal}
Dalam menganalisis kebutuhan IT Student Portal, penulis melakukan wawancara dengan 18 mahasiswa Program Studi Teknik Informatika UNPAR. Kriteria dari 18 mahasiswa tersebut yaitu lipsum. Setelah melakukan wawancara, penulis memperoleh fitur-fitur yang diinginkan mahasiswa antara lain:
\begin{enumerate}
	\item Prasyarat mata kuliah
	\item Status perkuliahan
	\item DPS dapat berubah sesuai riwayat nilai
	\item Susunan jadwal terurut
	\item Detail Kuliah
	\item Tampilan \textit{desktop} pada sistem operasi selain Windows 
	\item Daftar email dosen
	\item Upload CV
\end{enumerate}

Fitur-fitur yang akan dipilih untuk diimplementasikan harus memenuhi kriteria:
\begin{itemize}
	\item Data yang dibutuhkan dapat diambil dari Portal Akademik Mahasiwa
	\item Fitur tidak tersedia di Portal Akademik Mahasiswa
	\item Fitur mendukung fungsi Portal Akademik Mahasiswa sebagai sumber informasi akademik
\end{itemize}

Berdasarkan kriteria di atas dan batas waktu pembangunan aplikasi, maka akan dipilih fitur-fitur sebagai berikut:
\begin{enumerate}
	\item Prasyarat mata kuliah
	\item Susunan jadwal yang terurut 
	\item Tampilan \textit{desktop} pada sistem operasi selain Windows
\end{enumerate}


\section{Analisis Komunikasi Portal Akademik Mahasiswa untuk Fitur IT Student Portal}
\subsection{Kasus \textit{Login}}
Di Portal Akademik Mahasiswa, mahasiswa dapat login dengan mengakses \url{https://studentportal.unpar.ac.id/} dan mengklik tombol \texttt{input#submit.login-button} (Gambar \ref{}). Saat tombol tersebut ditekan, mahasiswa akan dibawa ke halaman \texttt{index.login.submit.php} dengan form data berisi \texttt{Submit: Login}. Lalu mahasiswa akan diarahkan ke \url{https://cas.unpar.ac.id/login?service=https\%3A\%2F\%2Fstudentportal.unpar.ac.id\%2Fhome\%2Findex.login.submit.php}. Di sana, mahasiswa akan ditampilkan halaman \textit{login} CAS UNPAR di mana mahasiswa diminta mengisi ``Username'' pada \textit{textfield} \texttt{input#username.required} dan mengisi ``Password'' pada \textit{password field} \texttt{input#password.required}. Setelah itu mahasiswa harus menekan tombol \texttt{input.btn-submit}. Data tersebut akan dikirimkan ke URL \url{} yang mengandung form data seperti pada gambar \ref{}. Jika berhasil, akan dilakukan pengalihan ke \url{https://studentportal.unpar.ac.id/main.php} dengan cookie.

\subsection{Kasus Nilai}
\subsection{Kasus Jadwal}
\subsection{Kasus \textit{Logout}}


%Student Portal UNPAR diakses dengan melakukan request pada URL \url{https://studentportal.unpar.ac.id/}. Saat mengklik tombol \textit{login}, halaman akan melakukan \textit{request} ke URL \url{https://studentportal.unpar.ac.id/home/index.login.submit.php} dengan mengirim Form Data \texttt{Submit=Login}. Halaman yang diperoleh dari pengiriman tersebut adalah halaman CAS UNPAR. Di halaman CAS UNPAR, \textit{login} akan dilakukan dengan mengirimkan data \texttt{username} yang berisi email \textit{student} UNPAR, \texttt{password} berisi kata sandi dari email \textit{student} UNPAR, \texttt{lt} diperoleh dari nilai elemen \texttt{input} dengan nama ``lt'', execution diperoleh dari nilai elemen \texttt{input} dengan nama ``execution'', dan \texttt{\_eventId} berisi ``submit"'. Setelah data tersebut dikirim ke URL \url{https://cas.unpar.ac.id/login}, akan diperoleh respon halaman depan Student Portal UNPAR dan \textit{cookies}.

