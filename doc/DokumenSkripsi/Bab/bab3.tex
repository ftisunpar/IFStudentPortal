\chapter{Analisis}
\label{chap:analisis}

\section{Analisis Fitur-fitur FTIS Student Portal}
\label{sec:fitur}

Berdasarkan penelitian, fitur-fitur yang diperlukan IT Student Portal adalah sebagai berikut:

\begin{enumerate}

\item Prasayarat Mata Kuliah \\
Masalah lain yang sering dialami FTIS adalah kesalahan dalam mengisi Formulir Rencana Studi(FRS). Meskipun sudah diberi daftar mata kuliah beserta prsyaratnya, mahasiswa sering kali lalai dalam memeriksa prasyarat tersebut. Selain itu, tidak semua dosen wali tahu prasyarat mata kuliah yang akan diambil sehingga dosen wali menyetujui mata kuliah yang masih belum boleh diambil. Akibatnya, banyak terjadi kesalahan dalam pengambilan mata kuliah. Kesalahan tersebut akan diperiksa oleh sekretaris jurusan. Artinya, sekretaris jurusan harus memeriksa kartu rencana studi setiap mahasiswa kemudian memeriksa prasyarat dari setiap mata kuliah yang diambil. 

Oleh karena itu, fitur prasyarat mata kuliah akan dibuat dalam IT Student Portal.  Prasyarat mata kuliah dapat berupa mata kuliah dan sks lulus. Jika pemeriksaan prasyarat mata kuliah dapat dilakukan oleh sistem, maka kesalahan pengambilan mata kuliah semakin berkurang. Selain itu, dosen wali dan sekretaris jurusan tidak perlu melakukan pemeriksaan lebih lanjut. 

\end{enumerate}