\chapter{Dasar Teori}
\label{chap:Dasar Teori}

\section{jsoup}
\label{sec:jsoup}

\textit{Web scraping}\cite{Vargiu:2013} adalah teknik mendapatkan informasi dari sebuah situs \textit{web} secara otomatis. Dalam Java, \textit{web scraping} dapat diimplementasikan menggunakan \textit{library} jsoup\cite{jsoup}. API yang disediakan oleh jsoup dapat digunakan untuk mengekstrak dan memanipulasi data HTML. 

Subbab-subbab berikut menjelaskan kelas-kelas dari jsoup yang terkait dengan pengerjaan skripsi ini.

\subsection{Jsoup}

Kelas ini merupakan inti untuk mengakses fungsi jsoup. Salah satu \textit{method} yang dimiliki kelas ini adalah sebagai berikut:
\begin{itemize}
	\item \textbf{public static Connection connect(String url)} \\
		Berfungsi untuk membuat koneksi baru dengan suatu situs \textit{web}. \\
		\textbf{Parameter:}
		\begin{itemize}
			\item \textbf{url} URL situs \textit{web} dengan protokol HTTP atau HTTPS.
		\end{itemize}
		\textbf{Kembalian:} koneksi dengan situs \textit{web}.
\end{itemize}

\subsection{Connection}

Kelas ini merupakan interface yang menyediakan pengambilan data dari situs \textit{web}. Beberapa \textit{method} yang dimiliki kelas ini adalah sebagai berikut:

\begin{itemize}
	\item \textbf{Connection cookies(Map<String,String> cookies)} \\
		Berfungsi untuk menambahkan \textit{cookie}. \\
		\textbf{Parameter:}
		\begin{itemize}
			\item \textbf{cookies} \textit{map} dari \textit{cookie}.
		\end{itemize}
		\textbf{Kembalian:} koneksi dengan situs \textit{web}.
		
		\item \textbf{Connection data(String key, String value)} \\
		Berfungsi untuk menambahkan parameter data. \\
		\textbf{Parameter:}
		\begin{itemize}
			\item \textbf{key} kunci data.
			\item \textbf{value} nilai data.
		\end{itemize}
		\textbf{Kembalian:} koneksi dengan situs \textit{web}.
		
		\item \textbf{Connection method(Connection.Method method)} \\
		Berfungsi untuk mengatur metode pengiriman. \\
		\textbf{Parameter:}
		\begin{itemize}
			\item \textbf{method} metode pengiriman HTTP \textit{request}.
		\end{itemize}
		\textbf{Kembalian:} koneksi dengan situs \textit{web}.
		
		\item \textbf{Connection timeout(int millis)} \\
		Berfungsi untuk mengatur batas waktu \textit{request}. Batas waktu nol akan dianggap sebagai batas waktu yang tak terhingga. \\
		\textbf{Parameter:}
		\begin{itemize}
			\item \textbf{millis} banyaknya milisekon sebelum batas waktu.
		\end{itemize}
		\textbf{Kembalian:} koneksi dengan situs \textit{web}.
		
		\item \textbf{Connection validateTLSCertificates(boolean value)} \\
		Berfungsi untuk mengatur pemeriksaan sertifikat TLS untuk HTTPS \textit{request}. Nilai "`true"' untuk memeriksa dan nilai "`false"' untuk tidak memeriksa.\\
		\textbf{Parameter:}
		\begin{itemize}
			\item \textbf{value} status pemeriksaan sertifikat TLS.
		\end{itemize}
		\textbf{Kembalian:} koneksi dengan situs \textit{web}.
		
		\item \textbf{Connection.Response execute()} \\
		Berfungsi untuk mengirim \textit{request}.\\
		\textbf{Kembalian:} objek Response.	
\end{itemize}

\subsection{Response}

Kelas ini merepresentasikan HTTP \textit{response}. Response merupakan turunan dari kelas Base yang menangani response dan \textit{request}. Beberapa \textit{method} yang dimiliki kelas ini adalah sebagai berikut:
\begin{itemize}
	\item \textbf{Map<String,String> cookies()} \\
		\textit{Method} ini diturunkan dari kelas Base, berfungsi untuk mendapatkan seluruh \textit{cookies}. \\
		\textbf{Kembalian:} seluruh \textit{cookies}.	
		
		\item \textbf{Document parse()} \\
		Berfungsi untuk \textit{parsing} \textit{response body} menjadi dokumen. \\
		\textbf{Kembalian:} koneksi dengan situs \textit{web}.
		
		\item \textbf{String body()} \\
		Berfungsi untuk mendapatkan \textit{response body} berupa \textit{string}. \\
		\textbf{Kembalian:} \textit{response body} dalam bentuk \textit{string}.
\end{itemize}

\subsection{Document}

Kelas ini merepresentasikan dokumen HTML. Salah satu \textit{method} yang dimiliki kelas ini adalah sebagai berikut:
\begin{itemize}
	\item \textbf{public Elements select(String cssQuery)} \\
		\textit{Method} ini diturunkan dari kelas Element, berfungsi untuk menemukan elemen HTML yang sesuai dengan kueri CSS. \\
		\textbf{Parameter:} 
		\begin{itemize}
			\item \textbf{cssQuery} kueri CSS.
		\end{itemize}
		\textbf{Kembalian:} elemen-elemen HTML yang sesuai dengan kueri CSS.	
\end{itemize}

\subsection{Elements}

Kelas ini merepresentasikan kumpulan elemen HTML. Beberapa \textit{method} yang dimiliki kelas ini adalah sebagai berikut:
\begin{itemize}
	\item \textbf{public Elements select(String query)} \\
		\textit{Method} Berfungsi untuk menemukan elemen-elemen yang sesuai dalam \textit{list} elemen. \\
		\textbf{Parameter:} 
		\begin{itemize}
			\item \textbf{query} kueri \textit{Selector}.
		\end{itemize}
		\textbf{Kembalian:} elemen-elemen yang sudah diseleksi sesuai kueri.	
		
		\item \textbf{public String val()} \\
		\textit{Method} Berfungsi untuk mendapatkan nilai dari elemen pertama. \\
		\textbf{Kembalian:} nilai elemen.	
		
		\item \textbf{public String text()} \\
		\textit{Method} Berfungsi untuk mendapatkan kombinasi teks dari seluruh elemen yang sesuai. \\
		\textbf{Kembalian:} seluruh teks dalam \textit{string}.	
\end{itemize}

\subsection{Element}

Kelas ini merepresentasikan sebuah elemen HTML yang berisikan \textit{tag}, atribut, dan \textit{child}. Beberapa \textit{method} yang dimiliki kelas ini adalah sebagai berikut:
\begin{itemize}
	\item \textbf{public Element child(int index)} \\
		\textit{Method} Berfungsi untuk mendapatkan \textit{child element} berdasarkan nomor indeks. \\
		\textbf{Parameter:} 
		\begin{itemize}
			\item \textbf{index} nomor index.
		\end{itemize}
		\textbf{Kembalian:} \textit{child element}.	
		
		\item \textbf{public Element child()} \\
		\textit{Method} Berfungsi untuk mendapatkan seluruh \textit{child element}. \\
		\textbf{Kembalian:} seluruh \textit{child element}.	
		
		\item \textbf{public String className()} \\
		\textit{Method} Berfungsi untuk mendapatkan nama kelas elemen. \\
		\textbf{Kembalian:} nama kelas elemen.	
		
		\item \textbf{public String text()} \\
		\textit{Method} Berfungsi untuk mendapatkan teks dari elemen. \\
		\textbf{Kembalian:} teks dalam \textit{string}.	
\end{itemize}



\section{Chrome DevTools}
\label{sec:devtools}

Chrome Developer Tools (DevTools) \cite{devtools} adalah perangkat \textit{debugging} yang dimiliki Google Chrome. Saat menunjungi suatu website, pengguna DevTools dapat melakukan debugging pada website tersebut. DevTools dapat diakses dengan menekan "`Ctrl+Shift+I"' saat sedang membuka suatu website.  

Fitur-fitur yang dimiliki DevTools antara lain:
\begin{enumerate}
	\item \textbf{Elements}, memeriksa dan mengubah elemen HTML dan \textit{style} dari suatu \textit{website}.
	\item \textbf{Console}, mendapatkan informasi pengembangan dan berinteraksi dengan dokumen.
	\item \textbf{Sources}, melakukan \textit{debugging} pada JavaScript dengan menentukan \textit{breakpoint}.
	\item \textbf{Network}, memantau kinerja jaringan pada \textit{website} secara \textit{real-time}.
	\item \textbf{Audits}, menganalisa halaman yang dimuat.
	\item \textbf{Timeline}, menampilkan alur waktu saat memuat halaman.
	\item \textbf{Profiles}, menggambarkan waktu eksekusi dan penggunaan memori saat memuat halaman.
	\item \textbf{Resources}, memeriksa sumber daya halaman yang dapat berupa basis data, \textit{cookies}, dan \textit{cache}.
\end{enumerate}

\begin{figure}[H]
	\centering
	\includegraphics[scale=0.5]{Gambar/chrome-devtools}
	\caption{Chrome DevTools} 
	\label{fig:chrome_devtools}
\end{figure}

Pada subbab-subbab berikut akan dijelaskan mengenai dua panel dari DevTools.
\subsection{Elements}
\subsection{Network}

\section{Play Framework}
\label{sec:play}

Play Framework \cite{Leroux:2014} merupakan sebuah web framework berbasis Java dan Scala. Play juga menggunakan \textit{design pattern} Model-View-Controller (MVC) di mana \textit{model} dan \textit{controller} menggunakan Java sedangkan \textit{view} menggunakan Scala dan HTML. 
