\chapter{Dasar Teori}
\label{chap:Dasar Teori}

\section{jsoup}
\label{sec:jsoup}

\textit{Web scraping} adalah teknik mendapatkan informasi dari sebuah \textit{website} secara otomatis. Dalam Java, \textit{web scraping} dapat diimplementasikan menggunakan \textit{library} jsoup \cite{jsoup}. API yang disediakan oleh jsoup dapat digunakan untuk mengekstrak dan memanipulasi data seperti HTML. 

Untuk mendapatkan data dari suatu \textit{website}, jsoup harus membuat koneksi terlebih dahulu dengan \textit{website} tersebut. Koneksi pada jsoup direpresentasikan oleh kelas Connection. \textit{Method} "`connect(String url)"' dapat dipanggil untuk membuat objek Connection baru dengan melempar IOException. Koneksi akan dibuat dengan mengirimkan HTTP \textit{request}. \textit{Method} "`connect(String url)"' merupakan \textit{static method} yang dimiliki oleh kelas Jsoup. Setelah membuat koneksi baru, objek Connection perlu memperhatikan pemanggilan \textit{method-method} berikut:
\begin{enumerate}
	\item \textbf{cookies()}, digunakan untuk menambahkan \textit{cookie} yang dikirim ke dalam \textit{request}.
	\item \textbf{data()}, digunakan untuk menambahkan parameter data ke dalam \textbf{request}.
	\item \textbf{timeout()}, digunakan untuk mengatur \textit{timeout request}.
	\item \textbf{validateTLSCertificates()}, digunakan untuk mengatur pemeriksaan sertifikat TLS untuk HTTPS \textit{request}. 
	\item \textbf{method()}, digunakan untuk mengatur metode pengiriman \textit{request}. 
	\item \textbf{execute()}, digunakan untuk mengirim \textit{request}. 
\end{enumerate}

Dengan pemanggilan \textit{method} execute(), jsoup akan mengirim \textit{request} ke website yang dituju. Kemudian jsoup akan menerima \textit{response} dari website tersebut. Dalam jsoup, HTTP \textit{response} direpresentasikan dalam kelas Response, maka HTTP \textit{response} yang diterima akan disimpan ke dalam objek Response yang kemudian akan dikembalikan oleh \textit{method} execute().

Response yang diperoleh akan di-\textit{parse} ke dalam bentuk \textit{Document Object Model} (DOM) yang direpresentasikan dalam kelas Document. Proses parsing dapat dilakukan dengan pemanggilan method parse() oleh objek Response. Document yang telah diperoleh dari hasil parsing dapat diseleksi untuk mendapatkan data yang diinginkan. Dalam menyeleksi Document, jsoup memanfaatkan CSS \textit{Selector} untuk mendapatkan elemen HTML yang dipanggil oleh objek Document dengan method select(). Hasil proses seleksi akan ditampung ke dalam objek bertipe Elements yang merepresentasikan elemen-elemen pada HTML. 

\section{Chrome DevTools}
\label{sec:devtools}

Chrome Developer Tools (DevTools) \cite{devtools} adalah perangkat \textit{debugging} yang dimiliki Google Chrome. Saat menunjungi suatu website, pengguna DevTools dapat melakukan debugging pada website tersebut. DevTools dapat diakses dengan menekan "`Ctrl+Shift+I"' saat sedang membuka suatu website.  

Fitur-fitur yang dimiliki DevTools antara lain:
\begin{enumerate}
	\item \textit{Elements}, memeriksa dan mengubah elemen HTML dan \textit{style} dari suatu \textit{website}.
	\item \textit{Console}, mendapatkan informasi pengembangan dan berinteraksi dengan dokumen.
	\item \textit{Sources}, melakukan \textit{debugging} pada JavaScript dengan menentukan \textit{breakpoint}.
	\item \textit{Network}, memantau kinerja jaringan pada \textit{website} secara \textit{real-time}.
	\item \textit{Audits}, menganalisa halaman yang dimuat.
	\item \textit{Timeline}, menampilkan alur waktu saat memuat halaman.
	\item \textit{Profiles}, menggambarkan waktu eksekusi dan penggunaan memori saat memuat halaman.
	\item \textit{Resources}, memeriksa sumber daya halaman yang dapat berupa basis data, \textit{cookies}, dan \textit{cache}.
\end{enumerate}

\begin{figure}
	\centering
	\includegraphics[scale=0.5]{Gambar/chrome-devtools}
	\caption{Chrome DevTools} 
	\label{fig:chrome_devtools}
\end{figure}



\section{Play Framework}
\label{sec:play}

Play Framework \cite{Leroux:2014} merupakan sebuah web framework berbasis Java dan Scala. Play juga menggunakan \textit{design pattern} Model-View-Controller (MVC) di mana \textit{model} dan \textit{controller} menggunakan Java sedangkan \textit{view} menggunakan Scala dan HTML. 
