\chapter{Pendahuluan}
\label{chap:pendahuluan}

\section{Latar Belakang}
\label{sec:latar_belakang}

Student Portal UNPAR[] merupakan sistem informasi berbasis \textit{web} yang digunakan oleh mahasiswa Universitas Katolik Parahyangan. Fitur-fitur yang dimiliki Student Portal UNPAR yaitu rencana studi, jadwal, nilai dan indeks prestasi, dan pembayaran uang kuliah. Namun, fitur-fitur tersebut masih belum cukup untuk mendukung kebutuhan akademik mahasiswa Program Studi Teknik Informatika. 

Salah satu fitur yang diperlukan oleh mahasiswa Teknik Informatika UNPAR adalah prasyarat mata kuliah. Dalam Teknik Informatika UNPAR, terdapat beberapa mata kuliah yang membutuhkan prasyarat baik prasyarat tempuh maupun prasyarat lulus. Misalkan dalam pengambilan mata kuliah AIF401 Skripsi 1 membutuhkan prasyarat lulus 108 sks dan lulus mata kuliah Penulisan Ilmiah. 

Untuk mendukung kebutuhan akademik mahasiswa Program Studi Teknik Informatika, fitur-fitur yang diperlukan akan dianalisa kemudian diimplementasikan ke dalam program IT Student Portal. Program yang akan dibuat merupakan program berbasis web menggunakan Play Framework. Selain itu, data-data yang akan ditampilkan diambil langsung dari Student Portal UNPAR dengan \textit{Web Scraping} menggunakan \textit{library} jsoup.

\section{Rumusan Masalah}
\label{sec:rumusan_masalah}

Rumusan dari masalah yang akan dibahas pada skripsi ini sebagai
berikut:
\begin{itemize}
	\item Fitur-fitur apa saja yang akan dibuat untuk IT Student Portal?
	\item Bagaimana mengimplementasikan \textit{Web Scraping} dengan \textit{library} jsoup?
\end{itemize}

\section{Tujuan}
\label{sec:tujuan}

Tujuan-tujuan yang hendak dicapai melalui penulisan skripsi ini sebagai berikut:
\begin{itemize}
	\item	Mengetahui fitur-fitur yang akan dibuat dalam IT Student Portal.
	\item	Mengimplementasikan \textit{Web Scraping} menggunakan \textit{library} jsoup.
\end{itemize}

\section{Batasan Masalah}
\label{sec:batasan_masalah}

Beberapa batasan yang dibuat terkait dengan pengerjaan skripsi ini sebagai berikut.
\begin{itemize}
	\item Aplikasi akan diuji pada server FTIS sehingga tidak bisa diakses dari luar.
\end{itemize}

\section{Metode Penelitian}
\label{sec:metode_penelitian}

Berikut ini adalah metode-metode yang dilakukan pada penelitian ini:

\begin{enumerate}
	\item Melakukan studi mengenai cara kerja Chrome DevTools.
  \item Melakukan studi mengenai \textit{library} jsoup.
	\item Menganalisa cara kerja Student Portal UNPAR.
	\item Mengimplementasikan \textit{library} jsoup
	\item Melakukan eksperimen dan pengujian.
\end{enumerate}

\section{Sistematika Penulisan}
\label{sec:sistematika_penulisan}

Sistematika penulisan setiap bab pada penelitian ini sebagai berikut:

\begin{enumerate}
  \item Bab Pendahuluan \\
  Bab 1 berisikan latar belakang, rumusan masalah, tujuan, metode penelitian,
  dan sistematika penulisan dari penelitian yang dilakukan.
  \item Bab Dasar Teori \\
  Bab 2 berisikan teori-teori yang menunjang penelitian yang dilakukan. Teori
  yang digunakan dalam penilitian ini, antara lain \textit{Web Scraping}, CSS \textit{Selector},
	\textit{Document Object Model}, Chrome DevTools, jsoup.
  \item Bab Analisis \\
  Bab 3 berisikan analisis yang dilakukan pada penelitian ini. Analisis yang
  dilakukan meliputi: Analisis Fitur-fitur FTIS Student Portal, Analisis \textit{Web Scraping}, 
	dan Analisis dari Aplikasi yang Akan Dibuat.
\end{enumerate}
